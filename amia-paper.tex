\documentclass[10.7pt,]{article}

\usepackage[letterpaper, margin=2.54cm, top=2.54cm]{geometry}
\usepackage[super,comma,sort&compress]{natbib}
\usepackage{lmodern}
\usepackage{authblk} % To add affiliations to authors
\usepackage{amssymb,amsmath}
\usepackage{wrapfig}
\usepackage{graphicx,grffile}
\usepackage[labelfont=bf,labelsep=period]{caption}
\usepackage{ifxetex,ifluatex}
%\usepackage{fixltx2e} % provides \textsubscript
\ifnum 0\ifxetex 1\fi\ifluatex 1\fi=0 % if pdftex
  \usepackage[T1]{fontenc}
  \usepackage[utf8]{inputenc}
\else % if luatex or xelatex
  \ifxetex
    \usepackage{mathspec}
  \else
    \usepackage{fontspec}
  \fi
  \defaultfontfeatures{Ligatures=TeX,Scale=MatchLowercase}
    \setmainfont[]{Arial Narrow}
    \setsansfont[]{Century Gothic}
    \setmonofont[Mapping=tex-ansi]{Consolas}
\fi
% use upquote if available, for straight quotes in verbatim environments
\IfFileExists{upquote.sty}{\usepackage{upquote}}{}
% use microtype if available
\IfFileExists{microtype.sty}{%
	\usepackage{microtype}
	\UseMicrotypeSet[protrusion]{basicmath} % disable protrusion for tt fonts
}{}

\usepackage{lipsum} % for dummy text only REMOVE

\newtheorem{exm}{Example}


%==============================
% Customization to make the output PDF 
% look similar to the MS Word version
%==============================
% To prevent hyphenation
\hyphenpenalty=10000
\exhyphenpenalty=10000

% To set the sections font size
\usepackage{sectsty}
\allsectionsfont{\fontsize{11}{11}\selectfont}
\sectionfont{\fontsize{14}{14}\selectfont}
\subsectionfont{\bfseries\fontsize{13}{13}\selectfont}
\subsubsectionfont{\bfseries\fontsize{11}{11}\selectfont}
%\subsubsectionfont{\normalfont}

% Spacing
\usepackage{setspace}
\usepackage{longtable}

\usepackage{xcolor}

% No new line after subsubsection
\makeatletter
%\renewcommand\subsubsection{\@startsection{subsubsection}{3}{\z@}%
%	{-3.25ex\@plus -1ex \@minus -.2ex}%
%    {-1.5ex \@plus -.2ex}% Formerly 1.5ex \@plus .2ex
%    {\normalfont}}
%\makeatother

\makeatletter % Reference list option change
\renewcommand\@biblabel[1]{#1.} % from [1] to 1
\makeatother %

% To set the doc title font
\usepackage{etoolbox}
\makeatletter
\patchcmd{\@maketitle}{\LARGE}{\bfseries\fontsize{15}{16}\selectfont}{}{}
\makeatother

% No page numbering
\pagenumbering{gobble}

\makeatletter
\def\maxwidth{\ifdim\Gin@nat@width>\linewidth\linewidth\else\Gin@nat@width\fi}
\def\maxheight{\ifdim\Gin@nat@height>\textheight\textheight\else\Gin@nat@height\fi}
\makeatother

% Scale images if necessary, so that they will not overflow the page
% margins by default, and it is still possible to overwrite the defaults
% using explicit options in \includegraphics[width, height, ...]{}
\setkeys{Gin}{width=\maxwidth,height=\maxheight,keepaspectratio}
\setlength{\parindent}{0pt}
\setlength{\parskip}{6pt plus 2pt minus 1pt}
\setlength{\emergencystretch}{3em}  % prevent overfull lines
\providecommand{\tightlist}{%
  \setlength{\itemsep}{0pt}\setlength{\parskip}{0pt}}
\setcounter{secnumdepth}{0}
% Redefines (sub)paragraphs to behave more like sections
\ifx\paragraph\undefined\else
\let\oldparagraph\paragraph
\renewcommand{\paragraph}[1]{\oldparagraph{#1}\mbox{}}
\fi
\ifx\subparagraph\undefined\else
\let\oldsubparagraph\subparagraph
\renewcommand{\subparagraph}[1]{\oldsubparagraph{#1}\mbox{}}
\fi
%==============================
\usepackage{hyperref}
\hypersetup{
	unicode=true,
	pdftitle={My Cool Title Here},
	pdfauthor={Author One, Author Two, Author Three},
	pdfkeywords={keyword1, keyword2},
	pdfborder={0 0 0},
	breaklinks=true
}
\urlstyle{same}  % don't use monospace font for urls

% Keywords command
\providecommand{\keywords}[1]
{
  \small	
  \textbf{Key words---} #1
}
%==============================

% reduce space between title and begining of page
\title{\vspace{-2em} Supplementary material for \\ Not So Weak-PICO: Leveraging weak supervision for Participants, Interventions, and Outcomes extraction for systematic review automation}
\date{\vspace{-5ex}}
\author[ ] {
    % Authors
    \bf\fontsize{13}{14}\selectfont
    Anjani Dhrangadhariya,\textsuperscript{\rm 1, 2}
    Henning M\"uller \textsuperscript{\rm 1, 2}
}
\affil[1]{Institute of Business Information Systems, University of Applied Sciences Western Switzerland (HES-SO Valais-Wallis), Sierre, Switzerland}
\affil[2]{Department of Computer Science, University of Geneva (UNIGE), Geneva, Switzerland}
\affil[*]{Corresponding author: Anjani Dhrangadhariya, Institute of Business Information Systems, University of Applied Sciences Western Switzerland (HES-SO Valais-Wallis), Sierre, Switzerland; anjani.dhrangadhariya@hevs.ch}
%==============================
\begin{document}
\maketitle
\vspace{2em} %separation between the affiliations and abstract
%==============================
\doublespacing
%==============================
%
%
%
%==============================
\section{Annotation guidelines: Study type and design entity}\label{lss}
%==============================
%
In the study type annotation, we intended to capture the following fine-grained information about the study type and design.

\begin{itemize}
    \item Whether the study participants were randomly allocated or not.
    \item Whether the study in question was a clinical trial, a case-control study or a clinical trial protocol.
    \item Preliminary information about masking should be also be included as in whether the study was single-blind, double-blind, triple-blind or quadruple blind. Do not include the information about how the masking was performed.
    \item Mark if a study was unblinded or open-label.
    \item Mark whether the study was a (placebo) controlled study i.e. either a control intervention or a placebo was used. If study is not described as a placebo controlled study but there is a mention of control intervention or control group, mark the generic terms.
    \item Mark whether a study was prospective or retrospective.
    \item If the study was performed at a single center mark information about mono-centric (mono-center) or multi-centric (multi-center).
    \item Clinical trials involving new drugs are commonly classified into five phases. Mark information about the trial phase.
    \item Information about whether a trial was non-inferiority, superiority, or equivalence trial
    \item Include other attributes like whether a study was cross-over, non-cross over or parallel, cluster or factorial trial design.
    \item Other information describing the trial design should be marked example pilot clinical trial, feasibility trial, trial protocol.
\end{itemize}
%
%
%
%==============================
\section{UMLS sources to PICO targets}\label{lss}
%==============================
%
\begin{longtable}{|l|p{0.3cm}|p{0.3cm}|p{0.3cm}|p{7.9cm}|}
    \hline
        Semantc Type & P & I & O & Reason for choosing \\ \hline
        Mental or Behavioral Dysfunction & 1 & -1 & 1 & Concepts under this semantic group can be participant (condition) dysfunction or an outcome endpoint (clinical manifestation) that could be measured. \\ \hline
        Sign or Symptom & 1 & -1 & 1 & Concepts under this semantic group could be the improvements in participant symptoms being measured as an outcome. \\ \hline
        Age Group & 1 & -1 & -1 & Concepts under this semantic group are relevant for participant age but are clearly not intervention or outcome concepts. \\ \hline
        Disease or Syndrome & 1 & -1 & -1 & Concepts under this semantic group are relevant for participant condition but are clearly not intervention or outcome concepts. \\ \hline
        Injury or Poisoning & 1 & -1 & -1 & Concepts under this semantic group are relevant for participant condition but are clearly not intervention or outcome concepts. \\ \hline
        Neoplastic Process & 1 & -1 & -1 & Concepts under this semantic group are relevant for participant condition but are clearly not intervention or outcome concepts. \\ \hline
        Patient or Disabled Group & 1 & -1 & -1 & Concepts under this semantic group are relevant for participant condition but are clearly not intervention or outcome concepts. \\ \hline
        Population Group & 1 & -1 & -1 & Concepts under this semantic group can include writing variations for participant gender, sex, social status and other characteristics. \\ \hline
        Acquired Abnormality & 1 & -1 & -1 & Concepts under this semantic group are relevant for participant condition but are clearly not intervention or outcome concepts. \\ \hline
        Anatomical Abnormality & 1 & -1 & -1 & Concepts under this semantic group are relevant for participant condition but are clearly not intervention or outcome concepts. \\ \hline
        Congenital Abnormality & 1 & -1 & -1 & Concepts under this semantic group are relevant for participant condition but are clearly not intervention or outcome concepts. \\ \hline
        Occupational Activity & 1 & -1 & -1 & Concepts under this semantic group could describe other relevant characteristics of the RCT participants. \\ \hline
        Professional or Occupational Group & 1 & -1 & -1 & Concepts under this semantic group could describe other relevant characteristics of the RCT participants. \\ \hline
        Geographic Area & 1 & -1 & -1 & Concepts in this semantic group can be related to ethnicity of a participant or the geographic location from where participants were enrolled \\ \hline
        Language & 1 & -1 & -1 & Concepts in this semantic group can be related to participants native language. \\ \hline
        Group & 1 & -1 & -1 & This semantic group is a parent node to all the other groups added under the participant class. \\ \hline
        Family Group & 1 & -1 & -1 & Could be RCT participant characteristics as well. I abstain on it because I am not sure what other kind of concepts could be encompass \\ \hline
        Group Attribute & 1 & -1 & -1 & Concepts in this semantic group can describe RCT participant groups like pregnant women, smokers, addicts, etc. \\ \hline
        Pathologic Function & 1 & -1 & -1 & Concepts under this semantic group are relevant for participant condition but are clearly not intervention or outcome concepts. \\ \hline
        Body Location or Region & 0 & 0 & 0 & ~ \\ \hline
        Body Part, Organ, or Organ Component & 0 & 0 & 0 & ~ \\ \hline
        Body Space or Junction & 0 & 0 & 0 & ~ \\ \hline
        Cell & 0 & 0 & 0 & ~ \\ \hline
        Cell Component & 0 & 0 & 0 & ~ \\ \hline
        Tissue & 0 & 0 & 0 & ~ \\ \hline
        Embryonic Structure & 0 & 0 & 0 & ~ \\ \hline
        Body System & 0 & 0 & 0 & ~ \\ \hline
        Fully Formed Anatomical Structure & 0 & 0 & 0 & ~ \\ \hline
        Anatomical Structure & 0 & 0 & 0 & ~ \\ \hline
        Receptor & 0 & 0 & 0 & ~ \\ \hline
        Animal & 0 & 0 & 0 & ~ \\ \hline
        Human & 0 & 0 & 0 & ~ \\ \hline
        Mammal & 0 & 0 & 0 & ~ \\ \hline
        Vertebrate & 0 & 0 & 0 & ~ \\ \hline
        Plant & 0 & 0 & 0 & ~ \\ \hline
        Research Activity & 0 & 0 & 0 & ~ \\ \hline
        Carbohydrate Sequence & 0 & 0 & 0 & ~ \\ \hline
        Classification & 0 & 0 & 0 & ~ \\ \hline
        Conceptual Entity & 0 & 0 & 0 & ~ \\ \hline
        Entity & 0 & 0 & 0 & ~ \\ \hline
        Environmental Effect of Humans & 0 & 0 & 0 & ~ \\ \hline
        Event & 0 & 0 & 0 & ~ \\ \hline
        Functional Concept & 0 & 0 & 0 & ~ \\ \hline
        Human-caused Phenomenon or Process & 0 & 0 & 0 & ~ \\ \hline
        Idea or Concept & 0 & 0 & 0 & ~ \\ \hline
        Machine Activity & 0 & 0 & 0 & ~ \\ \hline
        Manufactured Object & 0 & 0 & 0 & ~ \\ \hline
        Molecular Function & 0 & 0 & 0 & ~ \\ \hline
        Molecular Sequence & 0 & 0 & 0 & ~ \\ \hline
        Natural Phenomenon or Process & 0 & 0 & 0 & ~ \\ \hline
        Nucleotide Sequence & 0 & 0 & 0 & ~ \\ \hline
        Organism & 0 & 0 & 0 & ~ \\ \hline
        Organization & 0 & 0 & 0 & ~ \\ \hline
        Phenomenon or Process & 0 & 0 & 0 & ~ \\ \hline
        Physical Object & 0 & 0 & 0 & ~ \\ \hline
        Spatial Concept & 0 & 0 & 0 & ~ \\ \hline
        Cell or Molecular Dysfunction & 0 & -1 & 0 & ~ \\ \hline
        Occupation or Discipline & 0 & -1 & 0 & ~ \\ \hline
        Organism Attribute & 0 & -1 & 0 & ~ \\ \hline
        Mental Process & 0 & -1 & 0 & ~ \\ \hline
        Organ or Tissue Function & 0 & -1 & 0 & ~ \\ \hline
        Physiologic Function & 0 & -1 & 0 & ~ \\ \hline
        Body Substance & 0 & -1 & 0 & ~ \\ \hline
        Organism Function & 0 & -1 & 0 & ~ \\ \hline
        Social Behavior & 0 & -1 & 0 & ~ \\ \hline
        Biologic Function & 0 & -1 & 0 & ~ \\ \hline
        Health Care Related Organization & 0 & 0 & -1 & ~ \\ \hline
        Inorganic Chemical & 0 & 0 & -1 & ~ \\ \hline
        Fungus & 0 & -1 & -1 & ~ \\ \hline
        Virus & 0 & -1 & -1 & ~ \\ \hline
        Bacterium & 0 & -1 & -1 & ~ \\ \hline
        Medical Device & -1 & 1 & 1 & The concepts under this semantic group can include medical devices used for diagnostic tests or could be the medical devices used to carry out an intervention. Adding this semantic group could increase recall. \\ \hline
        Enzyme & -1 & 1 & 1 & Concepts under this semantic group could be a part of enzyme therapy interventions. Enzymes and related proteins are essential biological molecules which could serve as disease biomarkers (outcomes). \\ \hline
        Hormone & -1 & 1 & 1 & Concepts under this semantic group could be a part of hormone therapy interventions. Hormones could also serve as disease biomarkers (outcomes). \\ \hline
        Finding & -1 & -1 & 1 & Concepts under this semantic group could be the subjective or objective outcome measures of a participant. For example, Smoking cessation behaviour (CUI:C2586081). \\ \hline
        Laboratory or Test Result & -1 & -1 & 1 & Concepts under this semantic group could be the subjective or objective outcome measures of a participant. \\ \hline
        Clinical Attribute & -1 & -1 & 1 & An observable or measurable clinical attribute can be a patient outcome. \\ \hline
        Qualitative Concept & -1 & -1 & 1 & Concept under this semantic group could include names of measurable or qualitative outcome concepts. \\ \hline
        Intellectual Product & -1 & -1 & 1 & Concepts under this semantic group could include the official names of outcome measurement scales or questionnaires which also constitute a part of this entity. \\ \hline
        Quantitative Concept & -1 & -1 & 1 & Concepts under this semantic group could include measurable outcome concepts. \\ \hline
        Behavior & -1 & -1 & 1 & Behaviour concepts could be measurable or immeasurable RCT outcomes. For example, Treatment Compliance (CUI:C4319828). \\ \hline
        Individual Behavior & -1 & -1 & 1 & Individual behaviour concepts could be measurable or immeasurable RCT outcomes. For example, Treatment Compliance (CUI:C4319828). \\ \hline
        Laboratory Procedure & -1 & 0 & 0 & ~ \\ \hline
        Gene or Genome & -1 & -1 & 0 & ~ \\ \hline
        Research Device & -1 & -1 & 0 & ~ \\ \hline
        Indicator, Reagent, or Diagnostic Aid & -1 & -1 & 0 & ~ \\ \hline
        Cell Function & -1 & -1 & 0 & ~ \\ \hline
        Biomedical or Dental Material & -1 & 1 & -1 & Concepts under this semantic group are a part of intervention class. \\ \hline
        Clinical Drug & -1 & 1 & -1 & Concepts under this semantic group are a part of intervention class. \\ \hline
        Pharmacologic Substance & -1 & 1 & -1 & Concepts under this semantic group are a part of intervention class. \\ \hline
        Therapeutic or Preventive Procedure & -1 & 1 & -1 & Concepts under this semantic group are a part of intervention class. \\ \hline
        Vitamin & -1 & 1 & -1 & Concepts under this semantic group could form a part of diet and supplement interventions. \\ \hline
        Health Care Activity & -1 & 1 & -1 & Concepts under this semantic group are related to activities relating to the care of the patients which can be used either as control intervention or even an intervention in RCTs. \\ \hline
        Daily or Recreational Activity & -1 & 1 & -1 & Concepts under this semantic group could form a part of physical or behavioral intervention or can be used as placebo controls. \\ \hline
        Educational Activity & -1 & 1 & -1 & Concepts under this semantic group could form a part of educational or behavioral interventions in the RCTs. \\ \hline
        Immunologic Factor & -1 & 1 & -1 & Concepts under this semantic group could include vaccine names. \\ \hline
        Diagnostic Procedure & -1 & 1 & -1 & Concepts under this semantic group could be interventions or diagnosis instruments used in diagnostic RCTs \\ \hline
        Antibiotic & -1 & 1 & -1 & Concepts under this semantic group are a part of intervention class. \\ \hline
        Chemical Viewed Functionally & -1 & 1 & -1 & Concepts under this semantic group could be active compounds or IMPs of intervention used in RCT. \\ \hline
        Biologically Active Substance & -1 & 0 & -1 & ~ \\ \hline
        Chemical Viewed Structurally & -1 & 0 & -1 & ~ \\ \hline
        Food & -1 & 0 & -1 & ~ \\ \hline
        Temporal Concept & -1 & 0 & -1 & ~ \\ \hline
        Self-help or Relief Organization & -1 & 0 & -1 & ~ \\ \hline
        Activity & -1 & 0 & -1 & ~ \\ \hline
        Substance & -1 & 0 & -1 & ~ \\ \hline
        Chemical & -1 & 0 & -1 & ~ \\ \hline
        Organic Chemical & -1 & 0 & -1 & ~ \\ \hline
        Nucleic Acid, Nucleoside, or Nucleotide & -1 & 0 & -1 & ~ \\ \hline
        Amino Acid, Peptide, or Protein & -1 & 0 & -1 & ~ \\ \hline
        Fish & -1 & 0 & -1 & ~ \\ \hline
        Amphibian & -1 & -1 & -1 & ~ \\ \hline
        Reptile & -1 & -1 & -1 & ~ \\ \hline
        Regulation or Law & -1 & -1 & -1 & ~ \\ \hline
        Bird & -1 & -1 & -1 & ~ \\ \hline
        Eukaryote & -1 & -1 & -1 & ~ \\ \hline
        Hazardous or Poisonous Substance & -1 & -1 & -1 & ~ \\ \hline
        Element, Ion, or Isotope & -1 & -1 & -1 & ~ \\ \hline
        Amino Acid Sequence & -1 & -1 & -1 & ~ \\ \hline
        Biomedical Occupation or Discipline & -1 & -1 & -1 & ~ \\ \hline
        Professional Society & -1 & -1 & -1 & ~ \\ \hline
        Archaeon & -1 & -1 & -1 & ~ \\ \hline
        Governmental or Regulatory Activity & -1 & -1 & -1 & ~ \\ \hline
        Genetic Function & -1 & -1 & -1 & ~ \\ \hline
        Experimental Model of Disease & -1 & -1 & -1 & ~ \\ \hline
        Drug Delivery Device & -1 & -1 & -1 & ~ \\ \hline
        Molecular Biology Research Technique & -1 & -1 & -1 & ~ \\ \hline
\end{longtable}


\iffalse
\begin{table}[!ht]
    \centering
    \begin{tabular}{|l|r|l|r|l|r|}
    \hline
    \multicolumn{2}{|c|}{Intervention} & \multicolumn{2}{|c|}{Participant} & \multicolumn{2}{|c|}{Outcome} \\
    \hline
    Error type & Count & Error type & Count & Error type & Count \\
    \hline
        Description inconsistency & 18 & Remain unannotated & 93 & Scale inconsistency & 85 \\
        Period/article & 23 & Period/article & 15 & Period/article & 48 \\ 
        Unmarked control & 59 & Extra info marked & 80 & Extra info marked & 58 \\ 
        Junk information & 225 & Junk information & 53 & Junk information & 30 \\ 
        Generic name & 120 & Generic name & 90 & Generic name & 85 \\ 
        Conjunction connector & 36 & Conjunction connector & 30 & Conjunction connector & 57 \\ 
        Repeated mention & 220 & Repeated mention & 213 & Repeated mention & 207 \\
        - & - & - & - & Outcomes not marked & 71 \\ 
        Total errors & 701 & Total errors & 574 & Total errors & 641 \\ 
        %Total tokens evaluated & 12960 & Total tokens evaluated & 12960 & Total tokens evaluated & 12960 \\
        \hline
        \multicolumn{3}{|l|}{Total tokens evaluated} & \multicolumn{3}{|c|}{12960}\\
        \hline
    \end{tabular}
    \caption{\label{tab:errordist} Error distribution in the analysed tokens of EBM-PICO corpus.}
\end{table}
\fi
%
%
%
%==============================
\section{Experimental Details}\label{params}
%==============================
%
The label model was initialized with a cardinality of two for positive and negative labels for each target class. 
The parameters used to tune the Snorkel label model are listed in Table~\ref{lm:params}.
GridSearch was used to tune the parameters restricting the model search to 50/300 hyperparameter space.
%
\begin{table}[ht]
\centering
\begin{tabular}{|l|l|}
\hline
Parameters               & Values                              \\
\hline
Learning rate           & 0.001, 0.0001                      \\
L2 regularization       & 0.001, 0.0001                      \\
Epochs                  & 50, 100, 200, 600, 700, 1000, 2000 \\
Precision               & 0.6, 0.7, 0.8, 0.9                 \\
Optimizer               & adamax, adam, sgd                  \\
Learning rate scheduler & constant                           \\
\hline
\end{tabular}
\caption{\label{lm:params} Paramters used to tune Snorkel's label model using GridSearch}
\end{table}
% 


All the NER experiments in this article were conducted in PyTorch and the models were trained for 15 epochs with a mini-batch size of 10 for training and 6 for evaluation.
The maximum sequence length was set to 510 because of the transformer's limit.
For weakly and fully supervised experiments, the EBM-PICO training set was divided, 80\% of the data was used for training and 20\% for development.
The [CLS] embeddings from the PubMedBERT layer were used as features of the input text.
PubMedBERT was fine-tuned by not freezing weights during the experiments.
ReLU was used as the activation function before feeding [CLS] outputs to the linear layer.
The gradients were clipped to 1.0 to mitigate the problem of exploding gradients.
Each experiment was carried out on a single Quadro RTX 6000 GPU without data and model parallelization.
Further parameters used to train the weakly supervised PubMedBERT model are listed in Table~\ref{ws:params}.
%
\begin{table}[ht]
\centering
\begin{tabular}{|l|l|}
\hline
Parameters               & Values                            \\
\hline
Learning rate           & 0.0005                             \\
Learning rate warmup.   & 0.1.                               \\
Epsilon                 & 0.00000001                         \\
Epochs                  & 15                                 \\
Max sequence length     & 510                                \\
Optimizer               & AdamW.                             \\
\hline
\end{tabular}
\caption{\label{ws:params} Paramters used to train PubMedBERT-linear model}
\end{table}
%
%
%
%==============================
%\bibliographystyle{vancouver}
%\bibliography{literature}
%==============================

\end{document}